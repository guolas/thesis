\chapter{Conclusions and Future Work}\label{ch:conclusions_and_future}

% ------------------------------------------------------------------------------
\section{Conclusions} \label{sec:conc}

This work has analyzed the downlink of a clustered cellular network using
cooperation among the \glspl{bs} in the cluster through \gls{bd}. \gls{bd} is
able to eliminate the interference within the cluster, but does not handle the
\gls{oci}. Under these conditions, an analytical expression has been derived for
the mean achievable rate per user, which allows for an analysis of the influence
of different parameters of the system, such as the \gls{snr}, the antenna
configuration, the path-loss, and the cluster size. The derivation requires some
approximations that are used in order to model the \gls{oci}, which is
oversimplified as Gaussian noise, or neglected completely, in other works in the
literature.

Several effects have been observed, some of which have opposing impact on the
mean achievable rate:

\begin{itemize}
   \item Increasing the size of the cluster brings a natural reduction in the
      interference coming from other clusters.
   \item The coordination gain saturates because of the path-loss.
   \item Additionally, the increasing size yields a reduction in the power
      available for each data stream of each user (when the uniform power
      allocation is used) due to the need to coordinate with an increasing
      number of users, some of which may be far away.
\end{itemize}

The most interesting result is that the gain due to the coordination does not
grow unboundedly with the size of the cluster, but there is a limit on the size
of the cluster which, once surpassed, the gain obtained is limited. Recall that
the size of the cluster impacts, negatively, directly on the complexity of the
cooperation processing. It is of interest, hence, to form clusters of reduced
size. The present thesis yields a range of seven to ten cells, for a wide range
of \gls{snr}, and for different antenna configurations.

Apart from the mean achievable rate, also fairness considerations have been
analyzed in the same scenario, with the same \gls{bd} transmission strategy. In
order to do so, the \gls{cdf} of the achievable rate per user is shown to be
very accurately approximated via a Gamma distribution. Also the effects of
several system parameters have been taken into account and their effect on the
fairness of the system studied. In particular, even though an optimal power
allocation enables to increase the mean achievable rate, the fairness is
seriously affected by the power allocation scheme, and a much simpler uniform
power allocation is able to deliver equitable rates among the users.

By comparing the results of \gls{bd} on a scenario such as the one under study
in this thesis, it is clear to see how \gls{bd} and other coordination
techniques are not able to perform as well as expected. That is the reason why,
after analyzing a scenario where the \gls{oci} plays a main role, a simple yet
effective strategy has been proposed in order to overcome the impairments
introduced by the \gls{oci}. The strategy, consisting on a transmission scheme
based on local decisions made by the users of the system, is able to eliminate
the pernicious impact of the \gls{oci} on the performance of coordination
techniques, such as \gls{bd}.

An interesting collateral effect of the decisions being local to the users, is
that it makes the strategy able to adapt to changes in the conditions of the
channel of the users. Additionally, the fairness of the system is also improved,
for the users experiencing the lowest rates are the ones most benefitted from
the strategy proposed.

Finally, the complexity of the method proposed is kept low through the use of a
low complexity user scheduling algorithm that prevents the network from needing
to calculate \gls{svd}, which may be costly, especially as the number of users
per cell increases.

% ------------------------------------------------------------------------------
\section{Future lines of research} \label{sec:future}

The outcome of this thesis is, in a few words, that using clustering is
advantageous if coordination is intended to be used in real world scenarios. And
in this line, the proposal of a simple transmission strategy, with a low
complexity user scheduling algorithm, opens the door for further improvements to
make it an alternative for real applications.

In particular, there is an aspect that requires deeper analysis, and this is the
local decisions made by the users. This decisions are made by comparison of a
metric with a given threshold. In the current work the threshold has been
obtained via simulations, and this seriously reduces its attractiveness for
actual implementations. An interesting future line of work is trying to find a
way to calculate this threshold analytically, or at least not relying on
simulations to obtain it.

During the whole work, the assumptions made, although more or less common in the
literature, may be a bit strong in some circunstances. For instance the most
stringent one is the assumption that the \gls{csi} knowledge is perfect in all
the \glspl{bs}. A study of the influence of having imperfect \gls{csi} would be
a natural extension of the present work. Another imperfection that has not been
considered in this thesis, and that may be of importance when considering
transmitters located in distant locations, is the synchronization among them.

The proposed transmission strategy already alleviates the feedback capacity
required, for users not requesting coordination do not need to feed back the
whole channel matrix. Despite that, beamforming has the inherent need for
sharing user data, and this poses a big burden on the backhaul network. The
study of smart and efficient ways of distributing this information is an
interesting aspect to research.
