\chapter{Analysis of the State of the Art}\label{ch:state_art}

This thesis deals with three main topics:

\begin{itemize}
   \item Interference Management.
   \item Clustering.
   \item Scheduling.
\end{itemize}

This chapter presents a brief literature review on these fields, in order to put
this thesis in perspective. Interference management represents the cornerstone
of this document, hence it will be more thoroughly treated.

% ------------------------------------------------------------------------------
\section{Interference management} \label{sec:sa_interf}

The bulk of the work of this thesis is around interference management and, in
particular, around \gls{bd}, which is just a concrete case of a much broader
field.

Interference management techniques can be organized according to the information
requiered for the process to work:

\begin{description}
    \item[Beamforming] In order to perform beamforming, both \gls{csi} and user
        data should be shared among the transmitters. \gls{bd} lies within this
        category.
    \item[Interference Alignment] User data is not required for interference
        alignment to work, but \gls{csi} is still needed.
    \item[Blind techniques] Finally there are theoretical alternatives which may
        enable the interference management without \emph{a priori} knowledge
        about \gls{csi} or user data.
\end{description}

% ------------------------------------------------------------------------------
\subsection{Beamforming} \label{ssec:sa_bf}

Most of the work about beamforming applied to cellular networks was initially
based on the adaptation of \gls{mumimo} processing techniques to the context of
multi-cell cooperation, \cite{gesbert10}.

In particular the \gls{bc} and \gls{mac} channels are of special interest,
because a coordinated cellular network closely resembles these two channels, on
the downlink and on the uplink respectively. In \cite{caire00} the achievable
rate region of the \gls{mimo} \gls{bc} is analyzed for the case of single
antenna receivers, and it is shown that the use of \gls{dpc} is required to
achieve such region. The scenario with multiple antenna receivers is studied in
\cite{yu01}, using the same strategy based on \gls{dpc}.

While these works studied the achievable rate region using a given strategy,
the actual capacity for the \gls{bc} was calculated in \cite{yu04} and extended 
in \cite{weingarten06}. This capacity was also calculated using the so called
\gls{mac}-\gls{bc} duality which was presented \cite{jindal01} for the
\gls{siso} case and in \cite{vishwanath02} and \cite{jindal04} for the multiple
antenna scenario.

This thesis focuses on \gls{bd}, which was introduced in \cite{spencer04}
which, in turn, has spawned extensive research \cite{spencer04b, yoo06,
gesbert07b, shen06, wiesel08, stankovic08}, just to name a few. \gls{bd} was
originally formulated for \gls{mumimo}, but as with other techniques, it has
been also used for cooperative cellular networks. In general, \gls{bd} is
inferior in terms of sum capacity to \gls{dpc}. Nevertheless, in the special
case where \gls{bd} is combined with the right user selection algorithms,
\gls{bd} can come close to the sum capacity of \gls{dpc} \cite{shen06},
\cite{yoo06}.

The existing results for the \gls{mac} apply directly to a multicell setup
\cite{jafar04}. Despite the \gls{mac}-\gls{bc} duality, the extension of
\gls{mimo} \gls{bc} results to the multicell world presents several problems,
being one of the most important the power constraints. In the case of a
traditional \gls{mimo} \gls{bc}, the multiple antennas are collocated at a
single transmitter and, therefore, the power constraint is only one (arguably,
this is not completely true, because the power used for the transmission may not
be shared among antennas) while for the multicell case, the different
transmitters do not share a power constraint, yielding a whole different problem
formulation.

Early approaches like \cite{shamai01} and \cite{schubert02} dealt with a
cellular downlink as if it were a classical \gls{bc}, from the information
theoretical point of view. The key for their analysis is the use of the rather
unrealistic assumption of a sum-power constraint. They also share the single
antenna transmitters and receivers scenario formulation, and they differ in the
objective function that is considered, sum-rate \cite{shamai01} and individual
\gls{sinr} constraints per user in \cite{schubert02}. A similar scheme to
\cite{shamai01, schubert02} is used in \cite{karakayali06b}, with a linear
precoder based on the LQ factorization together with \gls{dpc} to handle the
remaining interference. The main difference in this work is that they consider a
\gls{mimo} scenario and, more importantly, the power constraint is set per base
station, which is a much more realistic assumption.

In \cite{venkatesan07} they propose the study of a realistic scenario in order
to analyze the advantages of ``Network \gls{mimo}'' when moderately realistic
assumptions are considered. The more realistic system model can be tackled
because they consider the uplink, where the assumptions needed to apply
conventional \gls{mimo} \gls{mac} strategies are simpler than for the downlink.

A first approach to using \gls{bd} in a multi-cell setup can be found in
\cite{shim08}, where no actual coordination exists among the different cells,
but they independently use \gls{bd} to serve their users. The key of this work
is that they consider a strategy that takes into account the interference coming
from other cells, in the form of the interference plus noise covariance matrix.
\cite{zhang09} goes one step further, by considering per-base-station power
constraint, and by considering clusters of coordinated \gls{bs} and partial
coordination among clusters. It also introduces a novel power allocation scheme,
\emph{scaled water-filling} which is used in this thesis for its simplicity and
good performance, despite being suboptimal. The per-base-station power
constraint is also considered in \cite{zhang10b}, \cite{liu09},
\cite{boccardi06}.

In \cite{shim08} they already proposed the use of limited size clusters, in
order to alleviate the burden put unto the backhaul network, feedback channels
and processing capabilities. But not only the complexity of the joint processing
increases with the size of the network, but research also shows that there may
be other limiting factors to the gains that can be achieved through coordination
\cite{lozano13}, \cite{barbieri12}. The first problem, the complexity increase,
can be handled by avoiding \gls{csi} sharing and through distributed cooperation
\cite{sun11}, \cite{bjornson10}, \cite{huang11}, \cite{bogale12}, \cite{he14}.
Another approach to deal with that problem is reducing the number of cells that
are coordinated, what is called as clustering. The research on this topic will
be reviewed in \refs{sec:sa_cluster}.

It is worth noting, albeit not studied within this thesis, that other linear
schemes based on minimizing the \gls{mse} have also been proposed, yielding
different precoding matrices \cite{shi08}, \cite{lu09}, \cite{armada11a}, \cite{shi11}, \cite{bogale12b}, \cite{sohn14}.

% ------------------------------------------------------------------------------
\subsection{Interference Alignment} \label{ssec:sa_ia}

The idea behind \gls{ia} is to find a way of restricting the interference
subspace to lie within a small subspace that does not span the entire signal
space at the receiver. This way, the desired signals can be projected into the
null space of the interference and, by that means, recovered free from
interference. The advantage of this is that, even though \gls{csi} is still
required to achieve that, there is no need to share user data, which is a big
reduction in the amount of information that is exchanged among the transmitters.

An exhaustive survey on \gls{ia} can be found in \cite{jafar10ianewlook}. The
initial works on this dealt with the \gls{mimo} X channel, with two users,
\cite{maddah06communication}, \cite{jafar08}, and it was extended to support
a bigger number of users in the K user interference channel \cite{cadambe08}.
This in turn was extended to the case of a \gls{mimo} interference channel
\cite{gou10}. A variety of increasingly sophisticated forms have arised from
these initial works, \cite{jafar14}, \cite{bresler14}, \cite{lashgari14} among
others.

In any case, \gls{ia} has found little use in real world applications, and it
remains as a very elegant and interesting theoretical field.

% ------------------------------------------------------------------------------
\subsection{Blind techniques} \label{ssec:sa_blind}

A further refinement over \gls{ia} is \gls{bia} \cite{gou11}, \cite{jafar14b}
and \cite{morales15}. In this case the channel coefficients are not required in
order to align the interference, and it is only needed to have information about
the coherence time of the channel, for these methods rely on the use of a
so-called supersymbol, during which the channel should stay unchanged. As with
regular \gls{ia}, the main drawback is the \gls{snr} regime required to perform
as expected, which usually exceeds the operation levels of real world cellular
networks.

% ------------------------------------------------------------------------------
\section{Clustering} \label{sec:sa_cluster}

As it has been said in \refc{ch:intro}, coordinating all the \glspl{bs} in the
system has drawbacks that can be alleviated reducing the number of elements that
coordinate. This can be achieved using clustering as a means to group a reduced
number of \glspl{bs} that will cooperatively process and transmit information to
the users.

Clustering can be done in several ways, and the particular characteristics of
an actual cluster formation may be highly dependent on the specific scenario to
be considered.

For analysis purposes the types of clusters can be grouped according to the
actual layout of the clusters, whether a given transmitter is allowed to belong
to more than one cluster at the same time or not:

\begin{itemize}
    \item Non-overlapping.
    \item Overlapping.
\end{itemize}

An alternative, non exclusive, characterization depends on the dynamic nature of
the clusters, whether they change over time or they stay fixed once they are
set:

\begin{itemize}
    \item Fixed.
    \item Dynamic.
\end{itemize}

In general, to make a certain clustering arrangement, it is necessary a metric,
a criterion to group several \glspl{bs}. Here there is a wide variety of choices
in the literature:

\begin{itemize}
    \item Proximity of the \glspl{bs}. 
    \item \gls{snr} of the link to each \gls{bs}.
    \item Level of interference coming from each \gls{bs}.
    \item Orthogonality of the channels.
    \item Transmission strategy design.
    \item Heuristic approaches.
\end{itemize}

The most common option, and simplest, is to have disjoint clusters, formed by
proximity of the \glspl{bs}, and fixed in time. The main advantage of this setup
is, precisely, its simplicity. Fixed clusters over time avoid the need for
reconfiguration, and the need for control information to be exchanged in order
to update the clusters configuration. This is the main reason why this is the
option used in this thesis. But it can also be found in many other references
\cite{zhang09}, \cite{huang09}, \cite{ramprashad10}, \cite{hosseini13}. This
advantage comes, in fact, from the fixed nature of the clusters, and not
necessarily from the distance metric, or the non-overlapping organization. In
\cite{swannack05}, fixed, non-overlapping clusters are also proposed, but the
metric used to build the clusters is more sophisticated. They select the
\glspl{bs} that have orthogonal channels to the users, which is advantageous
when \gls{zf} is used for coordination within each cluster.

The other alternative, is to have dynamic clusters, whose configuration may
change in time, evolving to meet the changing nature of the channel. The most
natural way of having dynamic clusters is when these groups of cells are
user-centric, \ie the clusters are formed around each user. If the user moves,
by definition, the clusters will vary accordingly. Another characteristic of
user-centric clusters is that they are overlapping. \cite{ng10} is an example of
such a system, where the \glspl{bs} that coordinate to transmit to each user are
assigned using a user-centric metric. In this case two different metrics are
proposed: distance based, and level of interference, intuitively it is more
likely to get a higher gain if the most interfering \glspl{bs} are coordinated.
A similar approach is used in \cite{gong11}, with overlapping, user-centric
clusters of fixed size, or of variable size according to the relative signal
strength received from each \glspl{bs}.

An interesting combination can be found in \cite{baracca12}, where user-centric
clusters are used, but where the resulting clusters are disjoint. The way to
achieve this is by carefully selecting which users are transmitted
simultaneously.

On the other hand, there may be dynamic clusters which are not user-centric.
This is the case of \cite{papadogiannis08}, \cite{papadogiannis10},
\cite{papadogiannis11}, \cite{moon11}, \cite{cui11}, \cite{lakshmana12}, and
\cite{mingyi13}. In general these approaches have in common that the clustering
is the by-product of another process, such as an optimization problem to reduce
the amount of feedback needed \cite{papadogiannis11}, \cite{lakshmana12}, or the
heuristic metrics used in \cite{moon11}.

In general, dynamic, overlapping clusters have the main drawback of requiring a
more complex management, requiring a higher amount of control information to set
them up and to update them.


% ------------------------------------------------------------------------------
\section{Scheduling} \label{sec:sa_sched}

In general, multi-user techniques can deal with just so many users in the system so that, when the number of users in too high, scheduling is required to ensure
that all users are served, in order to keep the fairness of the system
controlled. In \cite{yoo06}, the users are selected according to the mutual
orthogonality, and they combine it with a classical round robin scheduling,
which guarantees a maximum delay, and with a proportional fair strategy, which
enforces more strongly the fairness at the expense of not guaranteeing the
maximum delay in the transmissions. In \cite{moon13}, a heuristic algorithm is
proposed in order to successively schedule the users, in order to enforce
fairness. 

A similar, yet different use can be found in \cite{papadogiannis08} and
\cite{kaviani10} where a dynamic scheduling algorithm is used in order to
improve the fairness of the system, by reducing the effect of the edge users in
the performance of the network.

More sophisticated scheduling alternatives have been studied in \cite{yi11} and
\cite{ko12}, with essentially the same objective of controlling the equity of
the system.

While most of the scheduling research relies on finding an equitable solution
for the users transmission, a different use is proposed in \cite{baracca12},
where the user scheduling is a tool in order to enforce the creation of disjoint
clusters, easing like that the management of such a clustered network.
