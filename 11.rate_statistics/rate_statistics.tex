\chapter[Rate Statistics]{Achievable Rate and Fairness in Coordinated Base
Station Transmission\footnote{The work shown in this chapter has been published
in \cite{corvaja14}.}}\label{ch:rate_statistics}

% ------------------------------------------------------------------------------
\section{Introduction}\label{sec:stats_intro}

\refc{ch:achiev_rates} deals with the evaluation of the mean achievable rate,
and the analysis of the influence of the cluster size and power assignment on
the maximization of the rate. This chapter in turn focuses on the analysis of
the \gls{qos} of the system, in terms of the fairness in the distribution of the
achievable rate among the users.

Although the mean achievable rate, as calculated in \refc{ch:achiev_rates} can
provide useful information, also the \gls{cdf} plays an important role when
designing fairness, and other \gls{qos} related, management strategies in
coordinated downlink networks.

In the following, it is shown that the statistics of the achievable rate are
almost perfectly represented by a Gamma distribution. Note that the Gamma
distribution arises in several contexts when considering non-negative random
variables whose value is determined by several joint distributions. For example,
in \cite{heath13} it is used to model the interference in an
interference-limited cellular network in order to simplify the calculation of
the success probability, \ie the complementary event of an outage, and
the ergodic rate. In \cite{alahmadi10} a Gamma distribution is used to model
composite fading channels. And in \cite{atapattu11} a mixture Gamma distribution
is proposed to describe the \gls{snr} of wireless channels, mainly composite
shadowing/fading channels, but also many other small-scale fading channels.

This chapter provides a characterization of the \gls{cdf} of the achievable rate
in a coordinated base station transmission with \gls{bd}, where the \glspl{bs}
are grouped in clusters and the interference is due to adjacent clusters.

Using the system model from \refc{ch:system_model}, including the power
assignment strategies described there, and the cluster model proposed in
\refs{sec:achiev_clust_net}, the dependence of the \gls{cdf} on the cluster
size, and the power assignment strategy is studied as well in the current
chapter.

% ------------------------------------------------------------------------------
\section{Rate Statistics}\label{sec:stats_rate_stats}
% ------------------------------------------------------------------------------
\subsection{Cumulative Distribution Function}\label{ssec:stats_cdf}

Recal the expression of the achievable rate when \gls{bd} is used to coordinate
the \glspl{bs} in the cluster

\begin{equation} \label{eq:stats_achiev_rate}
    R_i^{\text{BD}}\left(d_i\right) = \sum\limits_{k = 1}^{\ell} \log_2 \left(
    1 + \frac{\lambdahat_{ik} p_{ik}}{\sigma_i^2 + I_i\left(d_i\right)} \right)
\end{equation}

\noindent
where the dependence on the $i$-th user's distance to its \gls{bs}, $d_i$, is
made explicit, the interference power is defined by \eqref{eq:interf_power}, the
noise power $\sigma_i^2$ is obtained as a function of the \gls{snr} using
\eqref{eq:snr_noise}, and the powers $p_{ik}$ would be computed using the power
allocation strategies described in \refs{sec:power_allocation}.

The derivation of the complete statistical characterization of the achievable
rate per user is an almost intractable task, due to the combination of many
effects such as the power assignment, the interference coming from outside the
cluster, the channel characteristics, etc. Even the evaluation of the mean
requires to resort to several approximations, although the final result has been
shown to be quite accurate, \emph{cf.} \refc{ch:achiev_rates}.

If the \gls{pdf} of the achievable rate per user is considered, a suitable
analytical model is provided by the Gamma distribution's \gls{pdf}

\begin{equation} \label{eq:gamma_pdf}
   f_{R}\left(x\right) = \frac{1}{\Gamma\left(\theta\right)\theta^k}x^{k-1}
   e^{-\frac{x}{\theta}}
\end{equation}

\noindent
with mean and variance given by $k\theta$ and $k\theta^2$, respectively, and
related to the system parameters as discussed in \refss{ssec:stats_mean} and
\refss{ssec:stats_variance}.

The choosing of the Gamma distribution is not arbitrary, and it is motivated by
the fact that the achievable rate per user in \eqref{eq:stats_achiev_rate} is
the result of adding several non-negative terms. The \emph{central limit theorem
for causal functions} \cite{papoulis_fourier} states that the convolution of an
unbounded number of causal functions can be approximated using a Gamma
distribution. A causal function is a function defined only for $\R^{+}
\cup \left\{0\right\}$, as $\log_2\left(1+x\right)$ with $x \geq 0$ is. In the
case under study the sum of terms is not unbounded, but \cite{papoulis_fourier}
shows also how the approximation is accurate also for a sum of a finite number
of terms.

The Gamma distribution has been introduced also in \cite{cheikh11} to describe
the \gls{sir} in a simpler environment, without noise and without any kind of
coordination.

In fact, when comparing the CDF obtained by simulation and a Gamma \gls{cdf},
with the same mean and variance, it is interesting to note a very accurate
fitting.

\begin{figure}[t]
\begin{center}
%     \dummybox
    \hspace*{8mm}\input ./11.rate_statistics/img/CDFgamma.tex
\end{center}
\caption{\gls{cdf} of the achievable rate per user with $t=r=2$ and
SNR=15\,dB. Comparison with the Gamma distribution with the same mean and
variance.}
\label{fig:cdf_sim_analy}
\end{figure}

\reff{fig:cdf_sim_analy} shows the perfect match between the experimental
\gls{cdf} and a Gamma \gls{cdf} for a uniform power allocation, and for
different system parameters.

The effect of increasing the size of the cluster on the \gls{cdf} of the
achievable rate is shown in \reff{fig:cdf_cluster_size}

\begin{figure}[t]
\begin{center}
%    \dummybox
    \hspace*{1mm}\input ./11.rate_statistics/img/CDFclusters.tex
\end{center}
\caption{\gls{cdf} of the achievable rate per user with $t=r=2$ and
SNR=15\,dB for different values of the cluster size.}
\label{fig:cdf_cluster_size}
\end{figure}

\reff{fig:cdf_cluster_size} also shows how the rate increases with the cluster
size up to a certain point, and then it starts to decrease. This is the same
behavior observed in the mean achievable rate in \refc{ch:achiev_rates}.

Another aspect that can be seen is how the variance of the rate decreases as the
cluster size increases.

% ------------------------------------------------------------------------------
\subsection{Effect of the power allocation}\label{ssec:stats_power_alloc}

Three different power allocation schemes have been considered:

\begin{itemize}
   \item Uniform power allocation, as described in
      \refss{ssec:uniform_allocation}.
   \item The modified water-filling from \refss{ssec:modified_wf}.
   \item Power allocation result of solving the convex optimization
      problem \eqref{eq:sum_rate_maxim} using numerical solvers. In particular
      the numerical solver used is CVX \cite{cvx}, \cite{gb08}. For the rest of
      the chapter, this solution will be noted as CVX solution.
\end{itemize}

\begin{figure}[t]
\begin{center}
%    \dummybox
    \hspace*{1mm}\input ./11.rate_statistics/img/CDFgamma_power.tex
\end{center}
\caption{\gls{cdf} of the achievable rate per user with $t=r=3$ and SNR=15\,dB.
Comparison between different power allocation schemes.}
\label{fig:cdf_power_alloc}
\end{figure}

In order to show the effect of using each of these power allocation schemes,
\reff{fig:cdf_power_alloc} presents the \gls{cdf} of the achievable rate per
user for the three of the schemes proposed, and using a constant antenna setup
of $t = r = 3$ antennas, an SNR=15\,dB, and two different cluster sizes, namely
$M \in \left\{5, 8\right\}$.

It should be noted that both the water-filling and the numerical solutions are
calculated for the problem formulated in \eqref{eq:sum_rate_maxim}, where no
out-of-cluster interference is present, so that these solutions are not adapted
to a multicluster environment. This is so because such adaptation would require
some sort of coordination among different clusters, and that is out of the scope
of this work.

% ------------------------------------------------------------------------------
\subsection{Mean value}\label{ssec:stats_mean}

The derivation of the mean value has been done in \refc{ch:achiev_rates}
resorting to some approximations that allow to evaluate the mean achievable rate
defined as in \eqref{eq:rate_bd_loc_aver}

\begin{equation} \label{eq:rate_bd_loc_aver_2}
    \bar{R}_i^{\text{BD}} = \int\limits_0^{\Rcell} R_i^{\text{BD}}\left(u
    \right) \frac{2u}{\Rcell^2} \dee u
\end{equation}

\noindent
using a particular model for the interference coming from outside the cluster,
\emph{cf.} \refs{sec:achiev_interf}.

The value of the mean that is thus obtained is rather accurate for a wide range
of the scenario parameters, as it is shown in detail in \refc{ch:achiev_rates}.

% ------------------------------------------------------------------------------
\subsection{Variance}\label{ssec:stats_variance}

A closed form expression for the variance of the achievable rate cannot be
obtained without too many simplifying approximations so, instead, an accurate
value is obtained through simulations.

\begin{figure}[t]
\begin{center}
%     \dummybox
    \hspace*{1mm}\input ./11.rate_statistics/img/Varianza.tex
\end{center}
\caption{Variance of the achievable rate per user with different values of
\gls{snr} and of $t$ and $r$ as a function of the cluster size $M$.}
\label{fig:rate_variance}
\end{figure}

\reff{fig:rate_variance} shows the variance of the achievable rate per user as a
function of the cluster size $M$. The different curves are for different system
parameters, for instance, several values of the \gls{snr} and different antenna
configurations. Other simulation parameters are fixed, as is the path loss
exponent $\gamma=3.8$, which is a rather typical value for urban environments as
used in \cite{karakayali06}, and the uniform power assignment that is
considered.

It is interesting to note that increasing the degrees of freedom available in
the system, by using a higher number of antennas, increases the variance of the
achievable rate. Something similar happens when the \gls{snr} of the system
increases.

An important detail that can be stressed is how the variance does not decrease
unboundedly with the cluster size, but a minimum can be found, analogously to
the maximum that can be found in the mean achievable rate in
\refss{ssec:achiev_optimum_size}.

Understanding that fairness in the system is closely related to the variance of
the achievable rates per user, it is plain to see that in order to maximize the
fairness of the system, a limited number of \glspl{bs} per cluster is
preferable.

% ------------------------------------------------------------------------------
\section{Fairness and QoS considerations}\label{sec:stats_fairnes_qos}

Using the statistics of the coordination schemes considered that have been
presented in \refs{sec:stats_rate_stats}, several fairness and \gls{qos}
characteristics can be studied.

One of these is the minimum rate that can be guaranteed to a percentage $X$ of
the users. This is given by the value of the rate, the abscise, when the
\gls{cdf} is equal to $\left(1 - \sfrac{X}{100}\right)$, which gives the rate
that can be guaranteed to $X$\,\% of the users.

\begin{figure}[t]
\begin{center}
%     \dummybox
    \hspace*{1mm}\input ./11.rate_statistics/img/Fairness90.tex
\end{center}
\caption{Rate achieved by 90\,\% of the users with different antenna
configurations and different values of \gls{snr}.}
\label{fig:min_rate_sys_param}
\end{figure}

\reff{fig:min_rate_sys_param} presents the rate achieved by 90\,\% of the users
as a function of the cluster size $M$, for different system parameters,
considering a uniform power assignment.

The behavior observed suggests, again, that a limited cluster size is advisable
in order to deliver the advantages of \gls{mimo} to the maximum number of users
possible.

A maximum of the minimum rate guaranteed to 90\,\% of the users appears for a
cluster size of around 7, and it decreases as the cluster grows bigger. This
decrease is steeper for higher number of antennas because the variance
increases, as seen in \reff{fig:rate_variance}. In particular, it can be seen
that for clusters of size 18, a configuration with $t=r=4$ performs worse than a
configuration with $t=r=3$.

Note, however, that the power assignment method plays an important role, as it
can be seen from the slope of the \gls{cdf} in \reff{fig:cdf_power_alloc}. In
order to illustrate this, \reff{fig:min_rate_power_alloc} represents the minimum
rate that can be guaranteed to 90\,\% of the users for different power
allocation strategies.

\begin{figure}[t]
\begin{center}
%     \dummybox
    \hspace*{1mm}\input ./11.rate_statistics/img/Fairness90power.tex
\end{center}
\caption{Rate achieved by 90\,\% of the users with different power assignment
schemes, $t=r=3$ and SNR$=15$\,dB.}
\label{fig:min_rate_power_alloc}
\end{figure}

The CVX power allocation scheme presents the worst performance, in terms of
the minimum guaranteed rate. It is easy to understand that the CVX solution
tries to maximize the sum-rate, which comes at the cost of letting the users
with the worst conditions out, yielding a very unfair system.

On the other hand, both waterfilling and the uniform assignment schemes perform
better in this regard, and they show a similar behavior. The only difference in
the behavior appears for bigger clusters. The uniform power assignment performs
better for small clusters, and the minimum rate decreases noticeably for large
clusters. The modified water-filling alternative shows a much more constant
behavior, with much less difference between small and big clusters.
