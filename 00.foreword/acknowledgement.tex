\chapter*{Agradecimientos}
\markright{Agradecimientos}

Mi primer y más importante agradecimiento va dirigido a mis padres. Por
respetar mis erráticos pasos por este mundo, incluso cuando no los compartían.
Siempre han estado ahí, como un contrafuerte sobre el que apoyar la carga de las
preocupaciones, cuando ha sido necesario. Mi hermana, en la distancia, también
ha sido participe de que sea como soy y quien soy ahora mismo. Ella pasó por lo
mismo pero, como siempre le ha ocurrido, sin un hermano mayor a quien recurrir,
haciéndolo más difícil. Gracias a ellos, llegar a este punto no ha sido tan
difícil como me gusta decir.

Y es que me he quejado mucho durante este tiempo, pero como dicen mis amigos los
pollos, ``he vivido como he querido''. En los últimos tiempos me he distanciado
de ellos, pero sé que siempre están ahí si realmente necesito algo de ellos, de
forma desinteresada, por ello quiero darles las gracias.

No me puedo olvidar, aunque lo había hecho, de Julio, José y Rober (y Bea cuando
se digna), que con esporádicos frikimiércoles han mantenido viva la chispa de la
curiosidad por cosas diferentes, cada uno hablándonos de nuestras cosas.

Por supuesto, con quien más tiempo he pasado, al final, ha sido con mis
compañeros de laboratorio, Máximo, Özge, Javier, Alex, Alex (el orden lo elegís
vosotros), Borja, Cecilia. En el fondo son la principal razón por la que merecía
la pena ir cada día a la universidad. Antes que todos ellos, Omar estuvo ahí, y
juntos compartimos penurias y lamentos, de los que espero que nos vayamos
librando poco a poco. A tita Sara tengo que agradecerle su optimismo resignado,
siempre enfrentándose a la dura realidad con una visión positiva. Lo mejor es
que ahora volveré a compartir tiempo con ella$\ldots$ un aliciente más para
afrontar el siguiente paso. Con un optimismo ligeramente diferente, Jair siempre
ha servido de motivación para seguir adelante. Su visión del mundo de la
universidad realmente llegaba a calar, y me hizo comprender, en ocasiones, que
no todo era tan negro como lo veía.

A mi tutora, Ana, he de agradecerle su optimismo infatigable y su capacidad para
ver lo bueno en todo, y por su paciencia aguantándome.

During the time I spent in the USA, I met very special people. First I want to
thank Nima and Tommi, because they were (and still are, from afar) true friends,
and made the life much easier in there. It is always a relief finding people who
are so similar to me, so far from home. Meo appeared at some point, too, with
her capacity to joke about almost anything, but still be serious when needed.
And she appeared through the most special person I met there, Samar. I spent
with her the most precious time I have spent with anyone in the recent years.
Despite the distance, she is still so important for me. I want to thank her
because she was the main reason to continue with the PhD, her selfless support
kept me focused and gave me strength to not give up. I want to truly thank her
for having so much faith in me.

Aparte de los compañeros de laboratorio, las actividades deportivas han sido
otro de los pilares que me han mantenido en pie. La natación y el aikido me han
dado la vida, a través de la salud. A mis monitores de natación (Jesús y Gema),
y a mi sensei (Julio) he de agradecerles sus esfuerzos. Y a mis compañeros de
aikido Borja, Gerardo, Javier y Rodolfo, agradecerles la posibilidad de entrenar
y crecer juntos.

Finalmente quiero agradecer el apoyo recibido de múltiples otros amigos y
amigas, que con el tiempo que me han dedicado han conseguido que pudiera
respirar a través de las brumas del desánimo. Unos más recientes, otros más
antiguos, su aportación ha sido igualmente importante: Carlos, Diego, Esther,
Isaac, Lydia, Marcos.
