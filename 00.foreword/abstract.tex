\chapter*{Abstract}

The need for higher data rates and higher efficiency in cellular networks
motivates the use of \gls{ufr}. Coordination among \glspl{bs} is required then
to alleviate the performance penalty due to the interference. Global
coordination is too complex and has inherent limitations that prevents it from
being used in real world scenarios. Clusters of a reduced number of \glspl{bs}
can be considered in order to ease off the requirements of coordination. As a
result, \gls{oci} appears, affecting negatively the communications.

This work focuses on \gls{bd}, a linear precoding technique that combines a
good theoretical performance with a relatively low complexity. However, the
unwanted interference seriously impacts the results obtained using \gls{bd}.

This thesis studies the downlink of a clustered cellular network, where \gls{bd}
is used to coordinate the \gls{bs} within each cluster. The mean achievable rate
is analyzed as a function of several scenario parameters. Of particular interest
is the dependence on the cluster size, which yields that there is an optimum
cluster size, beyond which no significant gain is obtained. Fairness
considerations are analyzed in the presence of \gls{oci}, also studied as a
function of scenario parameters such as the power allocation.

A mixed strategy using \gls{bd} and \gls{su} processing is proposed as a means
to overcome the impairment of the unhandled interference. The transmission
consists of two stages:

\begin{itemize}
    \item Users locally decide which transmission strategy they prefer and send
        this information to the \glspl{bs}.
    \item \glspl{bs} use the decisions of all users to schedule them for
        transmission so that the performance of the network is optimized.
\end{itemize}

The result of the proposed strategy is an improvement in the performance of
\gls{bd} in the presence of \gls{oci}, especially for the users experiencing the
worst conditions. This means that the fairness of the system is also increased,
along with the overall performance of the network.
