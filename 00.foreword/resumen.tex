\chapter*{Resumen}

La necesidad de tasas de transmisión más elevadas y una mayor eficiencia en las
redes celulares es la principal motivación para considerar el uso de \gls{ufr}.
La coordinación entre \glspl{bs} se hace necesaria, entonces, para compensar los
problemas introducidos por la interferencia. La coordinación global de la red es
demasiado compleja y, además, presenta limitaciones intrínsecas, que impiden su
utilización en escenarios reales. La utilización de grupos reducidos de
\glspl{bs} es una alternativa para reducir los requisitos impuestos por la
coordinación. Como consecuencia de la agrupación, aparece \gls{oci}, la cual
perjudica seriamente las comunicaciones.

Este trabajo se centra en \gls{bd}, una técnica de precodificación lineal que
combina unas buenas prestaciones con una complejidad relativamente baja. Sin
embargo, la interferencia empeora notablemente su funcionamiento.

En esta tesis se estudia el canal descendente de una red celular conglomerada,
donde se usa \gls{bd} para coordinar las \glspl{bs} que forman cada grupo. Se
analiza la tasa media obtenible como función de múltiples parámetros del
escenario. De especial interés es la dependencia con el tamaño de las
agrupaciones, de donde se desprende que existe un tamaño óptimo para los grupos
de \glspl{bs}, por encima del cual no se obtienen mejoras significativas. La
equidad del sistema se estudia en presencia de \gls{oci}, también como función
de diversos parámetros del escenario, como puede ser la asignación de potencia.

Se propone una estrategia mixta de transmisión, que combina \gls{bd} con
procesado \gls{su}, como mecanismo para combatir las dificultades introducidas
por la interferencia que no se gestiona. La estrategia consiste en dos fases:

\begin{itemize}
    \item Los usuarios deciden localmente qué estrategia prefieren para la
        transmisión, y envían esta información a las \glspl{bs}.
    \item Las \glspl{bs} utilizan las decisiones recibidas para planificar las
        transmisiones, de modo que se pueda optimizar el funcionamiento de la
        red.
\end{itemize}

El resultado de la estrategia propuesta es una mejora de las prestaciones de
\gls{bd} en presencia de \gls{oci}, especialmente para los usuarios más
desfavorecidos. Esto se traduce en que, adicionalmente, el sistema se vuelve más
justo, al mismo tiempo que el rendimiento de la red aumenta.
